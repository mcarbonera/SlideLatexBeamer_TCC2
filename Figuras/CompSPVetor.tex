\newcommand{\desenharLinhasParedeC}[2]{
	\begin{scope}[shift={(-0.38,0.6)},rotate=90]
		\begin{scope}[shift={(0,2-1.3)}]
			\node[inner sep = 0pt] (P1) at (0,0) {};
		\end{scope}
	\end{scope}
	\begin{scope}[shift={(-0.28,0.77)},rotate=45]
		\begin{scope}[shift={(0,2)}]
			\node[inner sep = 0pt] (P2) at (0,0) {};
		\end{scope}
	\end{scope}
	
	\draw (P1) -- (P2);
	
	% linhas continuacao:
	\def\UserVetAX{-1.08}
	\def\UserVetAY{0.6}
	
	\def\UserVetXNum{\fpeval{-0.28-sqrt(2)+1.08}}
	\def\UserVetYNum{\fpeval{0.77+sqrt(2)-0.6}}
	\def\UserVetMod{\fpeval{sqrt(\UserVetXNum*\UserVetXNum + \UserVetYNum*\UserVetYNum)}}
	
	\def\UserVetX{\fpeval{\UserVetXNum/\UserVetMod}}
	\def\UserVetY{\fpeval{\UserVetYNum/\UserVetMod}}
	
	\def\UserLinhaTrasX{\fpeval{-1.08 -\UserVetX*#1}}
	\def\UserLinhaTrasY{\fpeval{0.6 -\UserVetY*#1}}
	\def\UserLinhaFrenteX{\fpeval{-0.28-sqrt(2) +\UserVetX*#2}}
	\def\UserLinhaFrenteY{\fpeval{0.77+sqrt(2) +\UserVetY*#2}}
			
	\draw [dashed] (P1) -- (\UserLinhaTrasX,\UserLinhaTrasY);
	\draw [dashed] (P2) -- (\UserLinhaFrenteX,\UserLinhaFrenteY);
	
	\def\distance{\fpeval{\UserVetX * \UserVetAX + \UserVetY * (\UserVetAY - 0.3)}}
	\begin{scope}[shift={(0,0.3)}]
		\def\UserVetPerpX{\fpeval{\UserVetAX-\distance*\UserVetX}}
		\def\UserVetPerpY{\fpeval{(\UserVetAY - 0.3)-\distance*\UserVetY}}
		\def\UserVetPerpNormX{\UserVetPerpX/(sqrt(\UserVetPerpX*\UserVetPerpX + \UserVetPerpY*\UserVetPerpY))}
		\def\UserVetPerpNormY{\UserVetPerpY/(sqrt(\UserVetPerpX*\UserVetPerpX + \UserVetPerpY*\UserVetPerpY))}
	
		\draw[-{Latex[length=2mm]}] (0,0) -- (P1) node [black,midway,yshift=-0.21cm,xshift=0.1cm] {$u_a$};
		\begin{scope}[shift={(P1)}]
			\draw[-{Latex[length=2mm]}] (0,0) -- (\UserVetX, \UserVetY) node [black,midway,xshift=-0.3cm,yshift=0.1cm] {$u_t$};
		\end{scope}
		\draw[-{Latex[length=2mm]}] (0,0) -- (\UserVetPerpX, \UserVetPerpY)
		node [black,midway,yshift=0.3cm,xshift=0.2] {$u_p$};

		\def\UserChavesX{\fpeval{\UserVetAX-\distance*\UserVetX}}
		\def\UserChavesY{\fpeval{(\UserVetAY - 0.3)-\distance*\UserVetY}}	
		\draw [decorate,decoration={brace,amplitude=10pt},xshift=-2pt,yshift=-2pt]
			(\UserChavesX, \UserChavesY) -- (\fpeval{-1.08},\fpeval{0.3}) 
			node [black,midway,xshift=0.6cm, yshift=0.1cm] {$d$};
			
		% angulo reto:
		\def\UserTamanhoQuad{0.15}
		\def\UserPontoEsqX{\fpeval{\UserVetPerpX - \UserTamanhoQuad*\UserVetPerpNormX}}
		\def\UserPontoEsqY{\fpeval{\UserVetPerpY - \UserTamanhoQuad*\UserVetPerpNormY}}
		\def\UserPontoDirX{\fpeval{\UserVetPerpX + \UserTamanhoQuad*\UserVetX}}
		\def\UserPontoDirY{\fpeval{\UserVetPerpY + \UserTamanhoQuad*\UserVetY}}
		\def\UserPontoBaixoX{\fpeval{\UserVetPerpX - \UserTamanhoQuad*\UserVetPerpNormX + \UserTamanhoQuad*\UserVetX}}
		\def\UserPontoBaixoY{\fpeval{\UserVetPerpY - \UserTamanhoQuad*\UserVetPerpNormY + \UserTamanhoQuad*\UserVetY}}
		
		\draw (\UserVetPerpX, \UserVetPerpY) -- (\UserPontoEsqX, \UserPontoEsqY);
		\draw (\UserVetPerpX, \UserVetPerpY) -- (\UserPontoDirX, \UserPontoDirY);
		\draw (\UserPontoEsqX, \UserPontoEsqY) -- (\UserPontoBaixoX, \UserPontoBaixoY);
		\draw (\UserPontoDirX, \UserPontoDirY) -- (\UserPontoBaixoX, \UserPontoBaixoY);
		
		\filldraw (\fpeval{\UserVetPerpX - \UserTamanhoQuad*\UserVetPerpNormX/2 + \UserTamanhoQuad*\UserVetX/2}, 
		\fpeval{\UserVetPerpY - \UserTamanhoQuad*\UserVetPerpNormY/2 + \UserTamanhoQuad*\UserVetY/2}) circle (0.5pt);
		
		% Ângulo vetor
		\begin{scope}[shift={(P1)}, rotate=180]
			\draw (0.3*\UserVetX,0.3*\UserVetY) arc (110:160:0.3);
			\node at (-0.35,0.25) {\footnotesize $\theta$};
		\end{scope}
	\end{scope}
}

\begin{figure}[ht]
	\centering%	
	\begin{tikzpicture}[scale = 1.1]%
		% Obstáculo
		%\parede{3}{2}
		% Robô
		\begin{scope}[shift={(-1.1,1.4)},rotate=180]
			\meuRoboLindaoCompSP
			\desenharSensoresTrianguloSPa
			\desenharLinhasParedeC{3}{2}
		\end{scope}
	\end{tikzpicture}%
\end{figure}