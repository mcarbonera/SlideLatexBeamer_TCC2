\newcommand{\parede}[2]{
	\node[inner sep = 0pt] (P1) at (0,0) {};
	\node[inner sep = 0pt] (P2) at (#2,0) {};
	\node[inner sep = 0pt] (P3) at (#2,0.5) {};
	\node[inner sep = 0pt] (P4) at (0.5,0.5) {};
	\node[inner sep = 0pt] (P5) at (0.5,#1) {};
	\node[inner sep = 0pt] (P6) at (0,#1) {};
	\draw[color = darkgray] (P1) -- (P2);
	\draw[color = darkgray] (P2) -- (P3);
	\draw[color = darkgray] (P3) -- (P4);
	\draw[color = darkgray] (P4) -- (P5);
	\draw[color = darkgray] (P5) -- (P6);
	\draw[color = darkgray] (P6) -- (P1);
}

\newcommand{\sensorVisaoVetorSp}[1]{
	\draw[->] (0,0) -- (0,#1);
}

\newcommand{\desenharSensoresTrianguloSPb}{
	\begin{scope}[shift={(0.38,0.6)},rotate=-90]
		\sensorVisaoTriangularSp{2-0.9}
	\end{scope}
	\begin{scope}[shift={(-0.38,0.6)},rotate=90]
		\sensorVisaoTriangularSp{2}
	\end{scope}
	\begin{scope}[shift={(0.28,0.77)},rotate=-45]
		\sensorVisaoTriangularSp{2-0.5}
	\end{scope}
	\begin{scope}[shift={(-0.28,0.77)},rotate=45]
		\sensorVisaoTriangularSp{2}
	\end{scope}	
	\begin{scope}[shift={(0,0.87)},rotate=0]
		\sensorVisaoTriangularSp{2-1}
	\end{scope}
}

\newcommand{\desenharLinhasParedeA}{
	\begin{scope}[shift={(-0.38,0.6)},rotate=90]
		\begin{scope}[shift={(0,2-1.3)}]
			\node[inner sep = 0pt] (P1) at (0,0) {};
		\end{scope}
	\end{scope}
	\begin{scope}[shift={(-0.28,0.77)},rotate=45]
		\begin{scope}[shift={(0,2)}]
			\node[inner sep = 0pt] (P2) at (0,0) {};
		\end{scope}
	\end{scope}
	
	\draw (P1) -- (P2);
}

\newcommand{\desenharLinhasParedeB}{
	\begin{scope}[shift={(0.38,0.6)},rotate=-90]
		\begin{scope}[shift={(0,2-0.9)}]
			\node[inner sep = 0pt] (P1) at (0,0) {};
		\end{scope}
	\end{scope}
	\begin{scope}[shift={(0,0.87)},rotate=0]
		\begin{scope}[shift={(0,2-1)}]
			\node[inner sep = 0pt] (P2) at (0,0) {};
		\end{scope}
	\end{scope}
	
	\draw (P1) -- (P2);
}

\begin{figure}[ht]
	\centering%
	\begin{subfigure}[t]{0.5\textwidth}%
		\centering%
		\begin{tikzpicture}[scale = 1.1]%
			% Obstáculo
			\parede{3}{2}
			% Robô
			\begin{scope}[shift={(-1.1,1.4)},rotate=180]
				\meuRoboLindaoCompSP
				\desenharSensoresTrianguloSPa
				\desenharLinhasParedeA
			\end{scope}
		\end{tikzpicture}%
		
		\caption{Parede subestimada}%
	\end{subfigure}%
	~
	\begin{subfigure}[t]{0.5\textwidth}%
		\centering%
		\begin{tikzpicture}[scale = 1.1]%
			% Obstáculo
			\begin{scope}[shift={(0,0)},xscale=-1,yscale=1]
				\parede{3}{4}
			\end{scope}
			% Robô
			\begin{scope}[shift={(-2.4,2)},rotate=-90]
				\meuRoboLindaoCompSP
				\desenharSensoresTrianguloSPb
				\desenharLinhasParedeB
			\end{scope}
		\end{tikzpicture}%
			
		\caption{Parede superestimada}%
	\end{subfigure}%
\end{figure}