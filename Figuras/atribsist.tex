\newcommand{\sysxy}[3]
{
	\draw[gray, -{latex}] (xyz cs:x=0) -- (xyz cs:x=#3) node[above] {x};
	\draw[gray, -{latex}] (xyz cs:y=0) -- (xyz cs:y=#3) node[above] {y};
	
	\node[align=center] at (1,-1) (ori) {$ #1_{#2} $};
	\draw[dashed, ->,help lines,shorten >=3pt] (ori) .. controls (0.3,-0.65) and
	(0.5,-0.5) ..
	(0,0,0);
}
%\draw [double distance=1.5mm, very thick] (0,0)--(3,0);
		 
\def\UserDefLb{10}
\def\UserDefLe{2}
\def\UserDefbraco{5}
\def\UserDefAngulo{45}
\def\UserCoordx{5.5}
\def\UserCoordy{3.5}

\newcommand{\AtribSys}[3]{%
	%Coordenadas dos vértices da base
	\def\UserBbx{\fpeval{\UserDefLb}}
	\def\UserBby{0}
	\def\UserBcx{\fpeval{\UserDefLb/2}}
	\def\UserBcy{\fpeval{\UserDefLb*sqrt(3)/2}}
	
	%Coordenadas dos vértices do efetuador
	\def\UserEbx{\fpeval{\fpeval{\UserDefLe*cosd(#3)} + #1}}
	\def\UserEby{\fpeval{#2 + \UserDefLe*sind(#3)}}
	\def\UserEcx{\fpeval{#1 + \UserDefLe*cosd(#3+60)}}
	\def\UserEcy{\fpeval{#2 + \UserDefLe*sind(#3+60)}}
	
	%Meios vetores (entre base e efetuador)
	\def\UserVetAx{\fpeval{#1/2}}
	\def\UserVetAy{\fpeval{#2/2}}
	\def\UserVetBx{\fpeval{(\UserEbx - \UserBbx)/2}}
	\def\UserVetBy{\fpeval{(\UserEby - \UserBby)/2}}
	\def\UserVetCx{\fpeval{(\UserEcx - \UserBcx)/2}}
	\def\UserVetCy{\fpeval{(\UserEcy - \UserBcy)/2}}
	
	%Modulo dos meios vetores
	\def\UserModA{\fpeval{sqrt(\UserVetAx*\UserVetAx + \UserVetAy*\UserVetAy)}}
	\def\UserModB{\fpeval{sqrt(\UserVetBx*\UserVetBx + \UserVetBy*\UserVetBy)}}
	\def\UserModC{\fpeval{sqrt(\UserVetCx*\UserVetCx + \UserVetCy*\UserVetCy)}}
	
	%Modulo do vetor perpendicular (não unitário)
	\def\UserModXA{\fpeval{sqrt(\UserDefbraco*\UserDefbraco-\UserModA*\UserModA)}}
	\def\UserModXB{\fpeval{sqrt(\UserDefbraco*\UserDefbraco-\UserModB*\UserModB)}}
	\def\UserModXC{\fpeval{sqrt(\UserDefbraco*\UserDefbraco-\UserModC*\UserModC)}}
	
	%vetor perpendicular (não unitario)
	\def\UserXAy{\fpeval{-(\UserVetAx*\UserModXA)/\UserModA}}
 	\def\UserXAx{\fpeval{(\UserVetAy*\UserModXA)/\UserModA}}
 	
 	\def\UserXBy{\fpeval{-(\UserVetBx*\UserModXB)/\UserModB}}
 	\def\UserXBx{\fpeval{(\UserVetBy*\UserModXB)/\UserModB}}
 	
 	\def\UserXCy{\fpeval{-(\UserVetCx*\UserModXC)/\UserModC}}
 	\def\UserXCx{\fpeval{(\UserVetCy*\UserModXC)/\UserModC}}
	
	% Definindo angulos theta A, B e C
	\def\tetaA{atan2(\SomavetoresAy,\SomavetoresAx))}
	\def\tetaB{atan2(\SomavetoresBy,\SomavetoresBx)}
	\def\tetaC{atan2(\SomavetoresCy,\SomavetoresCx))+360}
	
	% Soma de vetores (braço + perpendicular)
	\def\SomavetoresAx{\fpeval{\UserVetAx + \UserXAx}}
	\def\SomavetoresAy{\fpeval{\UserVetAy + \UserXAy}}
	\def\SomavetoresBx{\fpeval{\UserVetBx + \UserXBx}}
	\def\SomavetoresBy{\fpeval{\UserVetBy + \UserXBy}}
	\def\SomavetoresCx{\fpeval{\UserVetCx + \UserXCx}}
	\def\SomavetoresCy{\fpeval{\UserVetCy + \UserXCy}}
	
	%Inicio da figura
	\node[fill,circle,inner sep=1.5pt] at (0,0) {};
	
	% Efetuador
	\begin{scope}[shift={(#1,#2)}, rotate = #3] 
		\node[fill,circle,inner sep=1.5pt] at
		(\UserDefLe/2,\fpeval{\UserDefLe*sqrt(3)/6}) (tool) {};
			
		\draw[-{latex},shorten >= 3.5pt] (tool) -- (0,0) node[below]
		{};
		\node[fill,circle,inner sep=1.5pt] at (0,0) {};
		\draw[-{latex},shorten >= 3.5pt] (tool) -- (\UserDefLe,0) node[below] {};
		\node[fill,circle,inner sep=1.5pt] at (\UserDefLe,0) {};
		\draw[-{latex},shorten >= 3.5pt] (tool) -- (\UserDefLe/2,
		\fpeval{\UserDefLe*sqrt(3)/2}) node[below] {};
		\node[fill,circle,inner sep=1.5pt] at (\UserDefLe/2,
		\fpeval{\UserDefLe*sqrt(3)/2}) {};
			
		\triangulo{\UserDefLe}
	\end{scope}
		
	% Flecha até tool
	\draw[-{latex},shorten >= 3.5pt] (0,0) -- (tool) node[below] {};
			
	% Braço 1
	\draw[-{latex},shorten >=3.5pt] (0,0) -- (\SomavetoresAx,\SomavetoresAy);
	\node[fill,circle,inner sep=1.5pt] at (\SomavetoresAx,\SomavetoresAy) {};
	\draw[-{latex},shorten >=3.5pt] (\SomavetoresAx,\SomavetoresAy) --
	 (#1, #2);
	
	% Braço 2
	\draw[-{latex},shorten >= 3.5pt] (\UserBbx, \UserBby) --
	+(\SomavetoresBx, \SomavetoresBy);
	\node[fill,circle,inner sep=1.5pt] at (\UserBbx + \SomavetoresBx, \UserBby +
	\SomavetoresBy) {};
	\draw[-{latex},shorten >= 3.5pt]
	(\UserBbx+\SomavetoresBx, \UserBby+\SomavetoresBy) -- (\UserEbx,\UserEby);
	
	% Braço 3
	\draw[-{latex},shorten >= 3.5pt] (\UserBcx, \UserBcy) --
		+(\SomavetoresCx, \SomavetoresCy);
	\node[fill,circle,inner sep=1.5pt] at (\UserBcx + \SomavetoresCx,
		\UserBcy + \SomavetoresCy) {};
		\draw[-{latex},shorten >= 3.5pt]
		(\UserBcx+\SomavetoresCx, \UserBcy+\SomavetoresCy) -- (\UserEcx,\UserEcy);
	 		
	% 1 - 2
	\node[fill,circle,inner sep=1.5pt] at (\UserDefLb,0) {};
	\draw[-{latex}, shorten >=3.5pt] (0,0) -- (\UserDefLb,0) node[below] {};
		
	% 1 - 3
	\begin{scope}[rotate=60]
		\node[fill,circle,inner sep=1.5pt] at (\UserDefLb,0) {};
		\draw[-{latex}, shorten >=3.5pt] (0,0) -- (\UserDefLb,0) node[below] {};
	\end{scope}
	
	%%%%% Sistemas de coordenadas
	%% Bi: 
	\sysxy{B}{1}{1} % B1
	\begin{scope}[shift={(\UserBbx,\UserBby)}] % B2
		\sysxy{B}{2}{1}
	\end{scope}
	
	\begin{scope}[shift={(\UserBcx,\UserBcy)}] % B3
		\sysxy{B}{3}{1}
	\end{scope}
	
	% Sistema O:
	%\begin{scope}[rotate=-30]
		\node[align=center] at (1.45,-1) (ori) {$, O$};
		%\draw[dashed, ->,help lines,shorten >=3pt] (ori) .. controls (0.3,-0.65) and
		%(0.5,-0.5) ..
		%(0,0,0);	
	%\end{scope}
	
	%% Mi:
	%%%%%%%%%%%%%% M2
	\draw[gray, -{latex}] (\SomavetoresAx,\SomavetoresAy) --
	+(\fpeval{\SomavetoresAx/(sqrt(\SomavetoresAx*\SomavetoresAx+\SomavetoresAy*\SomavetoresAy))},
	\fpeval{\SomavetoresAy/sqrt(\SomavetoresAx*\SomavetoresAx+\SomavetoresAy*\SomavetoresAy)})
	node[above] {x};
	
	\draw[gray, -{latex}] (\SomavetoresAx,\SomavetoresAy) --
	+(\fpeval{-\SomavetoresAy/(sqrt(\SomavetoresAx*\SomavetoresAx+\SomavetoresAy*\SomavetoresAy))},
	\fpeval{\SomavetoresAx/sqrt(\SomavetoresAx*\SomavetoresAx+\SomavetoresAy*\SomavetoresAy)})
	node[right] {y};
	
	\begin{scope}[shift={(\SomavetoresAx,\SomavetoresAy)}]
		\node[align=center] at (1,-1) (ori) {$ M_{1} $};
		\draw[dashed, ->,help lines,shorten >=3pt] (ori) .. controls (0.3,-0.65) and
		(0.5,-0.5) ..
		(0,0,0);
	\end{scope}
	
	%%%%%%%%%%%%%% M2
	\begin{scope}[shift={(\UserBbx,\UserBby)}]	
		\draw[gray, -{latex}] (\SomavetoresBx,\SomavetoresBy) --
		+(\fpeval{\SomavetoresBx/(sqrt(\SomavetoresBx*\SomavetoresBx+\SomavetoresBy*\SomavetoresBy))},
		\fpeval{\SomavetoresBy/sqrt(\SomavetoresBx*\SomavetoresBx+\SomavetoresBy*\SomavetoresBy)})
		node[above] {x};
		
		\draw[gray, -{latex}] (\SomavetoresBx,\SomavetoresBy) --
		+(\fpeval{-\SomavetoresBy/(sqrt(\SomavetoresBx*\SomavetoresBx+\SomavetoresBy*\SomavetoresBy))},
		\fpeval{\SomavetoresBx/sqrt(\SomavetoresBx*\SomavetoresBx+\SomavetoresBy*\SomavetoresBy)})
		node[above] {y};
		
		\begin{scope}[shift={(\SomavetoresBx,\SomavetoresBy)}]
			\node[align=center] at (1,-1) (ori) {$ M_{2} $};
			\draw[dashed, ->,help lines,shorten >=3pt] (ori) .. controls (0.3,-0.65) and
			(0.5,-0.5) ..
			(0,0,0);
		\end{scope}
	\end{scope}
	
	%%%%%%%%%%%%%% M3
	\begin{scope}[shift={(\UserBcx,\UserBcy)}]	
		\draw[gray, -{latex}] (\SomavetoresCx,\SomavetoresCy) --
		+(\fpeval{\SomavetoresCx/(sqrt(\SomavetoresCx*\SomavetoresCx+\SomavetoresCy*\SomavetoresCy))},
		\fpeval{\SomavetoresCy/sqrt(\SomavetoresCx*\SomavetoresCx+\SomavetoresCy*\SomavetoresCy)})
		node[above] {x};
		
		\draw[gray, -{latex}] (\SomavetoresCx,\SomavetoresCy) --
		+(\fpeval{-\SomavetoresCy/(sqrt(\SomavetoresCx*\SomavetoresCx+\SomavetoresCy*\SomavetoresCy))},
		\fpeval{\SomavetoresCx/sqrt(\SomavetoresCx*\SomavetoresCx+\SomavetoresCy*\SomavetoresCy)})
		node[below] {y};
		
		\begin{scope}[shift={(\SomavetoresCx,\SomavetoresCy)}]
			\node[align=center] at (1,-1) (ori) {$ M_{3} $};
			\draw[dashed, ->,help lines,shorten >=3pt] (ori) .. controls (0.3,-0.65) and
			(0.5,-0.5) ..
			(0,0,0);
		\end{scope}
	\end{scope}
	
	% Ei
	\begin{scope}[shift={(\UserCoordx,\UserCoordy)}]
		\node[align=center] at (1,-1) (ori) {$ E_{1} $};
		\draw[dashed, ->,help lines,shorten >=3pt] (ori) .. controls (0.3,-0.65) and
		(0.5,-0.5) ..
		(0,0,0);
	\end{scope}
	\begin{scope}[shift={(\UserEbx,\UserEby)}]
		\node[align=center] at (1,-1) (ori) {$ E_{2} $};
		\draw[dashed, ->,help lines,shorten >=3pt] (ori) .. controls (0.3,-0.65) and
		(0.5,-0.5) ..
		(0,0,0);
	\end{scope}
	\begin{scope}[shift={(\UserEcx,\UserEcy)}, rotate=90]
		\node[align=center] at (1,-1) (ori) {$ E_{3} $};
		\draw[dashed, ->,help lines,shorten >=3pt] (ori) .. controls (0.3,-0.65) and
		(0.5,-0.5) ..
		(0,0,0);
	\end{scope}
	
	\begin{scope}[shift={(#1,#2)}, rotate=#3]
		\begin{scope}[shift={(\UserDefLe/2,\fpeval{\UserDefLe*sqrt(3)/6})}]
			\draw[gray, -{latex}] (xyz cs:x=0) -- (xyz cs:x=0.6) node[above] {x};
			\draw[gray, -{latex}] (xyz cs:y=0) -- (xyz cs:y=0.6) node[above] {y};
			
			\begin{scope}[rotate=-45]
				\node[align=center] at (1,-1) (ori) {$W$};
				\draw[dashed, ->,help lines,shorten >=3pt] (ori) .. controls (0.3,-0.65) and
				(0.5,-0.5) ..
				(0,0,0);		
			\end{scope}			
		\end{scope}
	\end{scope}
	
	%% Angulos
	% Theta
	\draw[-latex, lightgray] (0:1.5) arc
	(0:\tetaA):1.5) node[right] at +(0.1,0.1) {$-\theta_{1}$};
	
	\begin{scope}[shift={(\UserBbx,\UserBby)}]
		%\draw[lightgray] (0.5,0) -- (0.9,0);
		\draw[-latex, lightgray] (0:0.7) arc (0:\tetaB:0.7)
		node[right] at +(0.1,0.1) {$\theta_{2}$};
	\end{scope}
	
	\begin{scope}[shift={(\UserBcx,\UserBcy)}]
		%\draw[lightgray] (1.3,0) -- (1.7,0);
		\draw[-latex, lightgray] (0:0.7) arc
		(0:\tetaC:0.7) node[right] at +(-0.6,0.8)
		{$\theta_{3}$};
	\end{scope}
	
	% Phi:
	\begin{scope}[shift={(\UserCoordx, \UserCoordy)}]
		\draw[lightgray] (0.3,0) -- (0.7,0);
		\draw[-latex, lightgray] (0:0.5) arc (0:#3:0.5) node[right] at +(0,-0.1)
		{$\phi$};
	\end{scope}
	
	% Psi
	\begin{scope}[shift={(\SomavetoresAx,\SomavetoresAy)}, rotate=\tetaA]
		%\draw[lightgray] (0.3,0) -- (0.7,0);
		\draw[-latex, lightgray] (0:0.5) arc
		(0:atan2(\UserCoordy-\SomavetoresAy,\UserCoordx-\SomavetoresAx)):0.7)
		node[right] at +(0.4,-0.3) {$\psi_{1}$};
	\end{scope}
	
	\begin{scope}[shift={(\UserBbx+\SomavetoresBx,\UserBby+\SomavetoresBy)},
	rotate = \tetaB]
		%\draw[lightgray] (0.3,0) -- (0.7,0);
		\draw[-latex, lightgray] (0:0.5)
		arc
		(0:atan2(\UserEby-\UserBby-\SomavetoresBy,\UserEbx-\UserBbx-\SomavetoresBx))-\tetaB:0.5)
		node[right] at +(0.55,0.35) {$\psi_{2}$};
	\end{scope}
	
	\begin{scope}[shift={(\UserBcx+\SomavetoresCx,\UserBcy+\SomavetoresCy)},
	rotate = \tetaC]
		%\draw[lightgray] (0.3,0) -- (0.7,0);
		\draw[-latex, lightgray] (0:0.5)
		arc
		(0:atan2(\UserEcy-\UserBcy-\SomavetoresCy,\UserEcx-\UserBcx-\SomavetoresCx))-\tetaC-360:0.5)
		node[right] at +(1.3,0.1) {$\psi_{3}$};
	\end{scope}
}

\begin{figure}[h]
\centering
	\begin{tikzpicture}[scale=0.85]
		\AtribSys{\UserCoordx}{\UserCoordy}{\UserDefAngulo}
	\end{tikzpicture}
\caption{Atribuição dos sistemas de coordenadas} \label{fig:atribsist}
\end{figure}