\tikzset{%
    block/.style={draw, fill=white, rectangle, 
            minimum height=2em, minimum width=3em},
    input/.style={inner sep=0pt},       
    output/.style={inner sep=0pt},      
    sum/.style = {draw, fill=white, circle, minimum size=2mm, node distance=1.5cm, inner sep=0pt},
    pinstyle/.style = {pin edge={to-,thin,black}}
}

\newcommand{\diagum}{
\begin{tikzpicture}[scale = 1,->,auto ,node distance =4 cm and 5cm , on grid,
>=latex' ,
state/.style ={scale = 0.9, circle, draw, minimum width =0.7cm},
finalstate/.style ={scale = 0.9, circle, draw, minimum width =0.7cm}]
	
	\node[state] (a) {0};
	\node[state, above right = 2.5cm of a] (b) {1};
	\node[state, below right = 2.5cm of b] (c) {2};
	\node[state, below left = 2.5cm of c] (d) {3};
	
	\path[->] (a) edge node[above left] {$\emptyset; a; c_1$} (b);
	\small
	\path[->] (b) edge node[above right] {$0<c_1<1; a; c_1$} (c);
	\normalsize
	\path[->] (c) edge node[below right] {$c_1<1; b; \emptyset$} (d);
	\draw (b) to [out=120,in=60,looseness=6] node[above] {$c_1\geq1;a;c_1$} (b);
	
	% Input
	\coordinate[left = 1cm of a, inner sep = 0pt] (input) {};
	\path[->] (input) edge (a);
\end{tikzpicture}%
}

\newcommand{\diagdois}{
\begin{tikzpicture}[->,auto ,node distance =4 cm and 5cm ,on grid ,
>=latex' ,
state/.style ={scale = 0.9, circle, draw, minimum width =0.7cm},
finalstate/.style ={scale = 0.9, circle, draw, minimum width =0.7cm}]

	\node[state] (a) {0};
	\node[state, above right = 2.5cm of a] (b) {1};
	\node[state, below right = 2.5cm of b, align = center] (c) {2 \\ $c_1<1$};
	\node[state, below left = 2.5cm of c] (d) {3};
	
	\path[->] (a) edge node[above left] {$\emptyset; a; c_1$} (b);
	\small
	\path[->] (b) edge node[above right] {$0<c_1<1; a; c_1$} (c);
	\normalsize
	\path[->] (c) edge node[below right] {$\emptyset; b; \emptyset$} (d);
	\draw (b) to [out=120,in=60,looseness=6] node[above] {$c_1\geq1;a;c_1$} (b);
	
	% Input
	\coordinate[left = 1cm of a, inner sep = 0pt] (input) {};
	\path[->] (input) edge (a);	
\end{tikzpicture}%
}

% Figura
\begin{figure}[h]
\centering
%\caption{Exemplos de diagramas de transição para autômatos temporizados com
%guarda} 
	%\label{fig:ATG}
	\begin{subfigure}[t]{0.5\textwidth}%
		\centering
		\diagum%
		%\label{fig:guardadiag1}%
		%\caption{Exemplo 1}%
		\end{subfigure}%
    	~
    	\begin{subfigure}[t]{0.5\textwidth}%
		\centering
		\diagdois%
		%\label{fig:guardadiag2}%
		%\caption{Exemplo 2}%
		\end{subfigure}%
		
		%\textbf{Fonte: baseado em \citeonline{book:SED}, p. 301.}
\end{figure}